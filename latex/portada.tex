%\vspace{-2.0cm}
\pagenumbering{roman}
\thispagestyle{empty}
\begin{center}
	UNIVERSIDAD CENTRAL DE VENEZUELA\\
	FACULTAD DE CIENCIAS\\
	POSTGRADO EN CIENCIAS DE LA COMPUTACI\'ON\\

	\begin{figure}
						\centering
						  \includegraphics[height=.7\textwidth]{images/UCV.png}
  \end{figure}
  \vspace{1.5cm}
  \large{\textbf{RECUPERACI\'ON, EXTRACCI\'ON Y CLASIFICACI\'ON DE \\ INFORMACI\'ON DE SABER UCV}}

  \vspace{3cm}
  Trabajo de grado de Maestría presentado ante la \\
  ilustre Universidad Central de Venezuela por el\\
  Econ. José Miguel Avendaño Infante para  optar
  al título de \\Magister Scientiarum en Ciencias de la Computaci\'on\\
  \vspace{0.5cm}
  Tutor: Dr. Andres Sanoja\\
  \vspace{1.5cm}
  Caracas - Venezuela\\
  Octubre 2023
\end{center}


\newpage
\thispagestyle{empty}
\large{\textbf{Resumen:}}

Se presenta la investigación e implementación de un sistema para hacer la \emph{Recuperación, Extracción y Clasificación de Información de SABER UCV}. El sistema ejecuta procesos de clasificación, almacenamiento y recuperación de información sobre los Resúmenes, tanto de las tesis como de los trabajos especiales de grado (TEG) que se encuentran publicados en el repositorio institucional Saber UCV y permite al usuario hacer procesos de búsqueda mediante una aplicación web.

A los Resúmenes con los cuales se conforma el corpus, se le aplican técnicas de Procesamiento de Lenguaje Natural (PLN), de Minería de Texto, de indexación en base de datos y se usan modelos preentrenados de inteligencia artificial para generar \textit{embeddings} y soportar procesos de búsqueda semántica.

Mediante la búsqueda de palabras o frases, aplicando filtros como fechas de publicación, área en la cual se generó la investigación y/o el nivel académico, el usuario genera un \textit{query} con el que se pueden recuperar los trabajos de mayor relevancia en que se encuentran contenidas tales palabras, o palabras similares y a partir de ahí enriquecer la experiencia del usuario con la presentación de la información extraída en tablas interactivas, gráficos, grafos de coocurrencia de palabras y recomendaciones de textos que puedan ser de interés para el investigador.

La aplicación se implementa como un sistema distribuido bajo la arquitectura cliente-servidor y se soporta en el uso de contenedores orquestados.

También se implementa una rutina que permite clasificar las tesis y los TEG, según el área académica donde cursó estudios el autor del correspondiente trabajo y así se solventa la carencia que actualmente presenta Saber UCV, donde no están disponibles estas clasificaciones.

\vspace*{2cm}

\textbf{Palabras Clave::} Recuperación de Información, Procesamiento del Lenguaje Natural,  Minería de Texto, Relevancia, Búsqueda de Texto Completo, Inteligencia Artificial, Embeddings, Búsqueda Semántica, Mapas del Conocimiento.



\newpage
\thispagestyle{empty}
\large{\textbf{Abstract:}}

The research and implementation of a system for the Retrieval, Extraction, and Classification of Information from SABER UCV (UCV's Institutional Repository) is presented. The system performs processes of classification, storage, and retrieval of information about Abstracts, both from Theses and Special Degree Works (TEG), that are published in the institutional repository Saber UCV, allowing users to perform search processes through a web application.

Natural Language Processing (NLP) techniques, Text Mining, database indexing, and pre-trained artificial intelligence models are applied to the Abstracts forming the corpus. These techniques are used to generate embeddings and support semantic search processes.

By searching for words or phrases and applying filters such as publication dates, research area, and/or academic level, the user generates a query with which the most relevant works containing those words or similar words can be retrieved. From there, the user experience is enhanced by presenting the extracted information in interactive tables, graphs, word co-occurrence graphs, and recommendations of texts that might be of interest to the researcher.

The application is implemented as a distributed system under the client-server architecture and relies on the use of orchestrated containers.

Additionally, a routine is implemented that allows the classification of Theses and TEGs based on the academic area where the corresponding work's author studied, thus addressing the current deficiency of Saber UCV, where these classifications are not available.

\vspace*{2cm}

\textbf{Keywords:} Information Retrieval, Natural Language Processing, Text Mining, Relevance, Full Text Search, Artificial Intelligence, Embeddings, Semantic Search, Knowledge Maps.


\thispagestyle{empty}


%\newpage


\setlength{\abovedisplayskip}{-5pt}
\setlength{\abovedisplayshortskip}{-5pt}
\thispagestyle{empty}

\newpage
\begin{center}
\large{\textbf{\emph{\Huge{Dedicatoria:}}}}
\end{center}
\thispagestyle{empty}
\vspace*{5cm}
\thispagestyle{empty}
\begin{center} \Large \emph{A  mi hijo Cassiel y  } \end{center}
\vspace*{1cm}
\begin{center} \Large \emph{mi esposa Waleska.} \end{center}



\newpage
\begin{center}
\large{\textbf{\emph{\Huge{Agradecimientos:}}}}
\end{center}
\thispagestyle{empty}
\vspace*{2cm}
\thispagestyle{empty}

- A mi madre, obvio, sino no habría ni una sola palabra acá.\\\\
- A mi padre Fernando por negarme el Atari e insistir en el Oddysey 2.\\\\
- A mi tía Mercedes Infante.\\\\
- A Cesar Alejandro García por todas las ayudas.\\\\
- A mi hermano David por su soporte.\\\\
-   Dr. Andres Sanoja primero por aceptar la tutoría y enseñarme qué es la investigación dentro de una comunidad científica.\\\\
-   Dr. José Mirabal por siempre andar con alguna idea a desarrollar y el tiempo dedicado a escuchar las propuestas y complementar esta Investigación.\\\\
-   Dra. Concettina Di Vasta por las imponentes sesiones de 2 horas 15 minutos llenas de coherencia y conocimiento.\\\\
-   Dra. Haydemar Nuñez por la rigurosidad al impartir los conocimientos.\\\\
-   Dra. Vanessa Leguizamo por tomarse el tiempo de revisar la solicitud de estudio de un oxidado economista y por ser mi Prof.ª.\\\\
-   Dra. Nuri Hurtado Villasana por tomarse el tiempo de escucharme y brindarme sugerencias en la elaboración de este trabajo.\\\\
-   A todo el personal del Postgrado: sus buenos días, por tener a mano la llave, por ayudar a mantener viva la Academia.\\\\
-   Mauricio Sáez Toro del equipo Saber UCV por mantener activo el Sistema Saber UCV y tener el tiempo de haber colaborado con esta investigación.\\\\
-   A toda la comunidad de creadores de software libre y open sourcem en especial a los \#useRs por motivarme a adentrarme al mundo de las ciencias de la computación.\\\\
- A Alexandra Asanovna Elbakyan, sin ella serían mínimas las citas bibliográficas de esta Investigación.






\newpage
\thispagestyle{empty}
\vspace*{5cm}
\hfill
\begin{minipage}{0.70\textwidth}
\begin{quote}
Como todos los hombres de la Biblioteca, he viajado en mi juventud; he peregrinado en busca de un libro, acaso del catálogo de catálogos; ahora que mis ojos caso no pueden descifrar lo que escribo, me preparo a morir a unas pocas leguas del hexágono en que nací.\\
--- Jorge Luis Borges, \textit{La Bibioloteca de Babel}, Ficciones
\end{quote}
\hspace*{2cm}

\begin{quote}
Every important aspect of programming arises somewhere in the context of sorting or searching.

--- Donald Knuth, \textit{The Art of Computer Programming}, Volume 3
\end{quote}
\end{minipage}

\thispagestyle{empty}
\maketitle


